\begin{document}

\begin{frame}
	 \titlepage
\end{frame}

\begin{frame}{Gliederung}
	\tableofcontents
\end{frame}

\section{Herrscher}

\subsection{Adolf Hitler -- Führer und Reichskanzler}

\begin{frame}{Adolf Hitler}
	\begin{columns}
		\begin{column}{0.6\textwidth}
			\includegraphicscopyright[height = 6cm]
				{Bundesarchiv_Bild_183-S33882,_Adolf_Hitler_retouched.jpg}
				{Bundesarchiv, Bild 183-S33882; CC-BY-SA}
		\end{column}
		\begin{column}{0.39\textwidth}
			\begin{itemize}
				\item 20. April 1889 - 30. April 1945
				\item Reichskanzler seit 1933
			\end{itemize}
		\end{column}
	\end{columns}
\end{frame}


\begin{frame}{Adolf Hitler, wenn er überlebt hätte}
	\begin{itemize}
		\item beging Selbstmord noch vor Kriegsende
		\pause
		\item wäre wohl angeklagt worden
		\pause
		\item Todesurteil
		\pause
		\item wenn nicht Todesurteil, lange Haftstrafe
		\item garantiert nicht weiter Staatsoberhaupt
	\end{itemize}
\end{frame}

\subsection[Der Ten’nou]{Der japanische Ten’nou (Kaiser)}

\begin{frame}{Kaiser Hirohito -- Shouwa-Ten’nou}
	\begin{columns}
		\begin{column}{0.6\textwidth}
			\includegraphicscopyright[height = 6cm]
			{Macarthur_hirohito.jpg}
			{U.S. Army photographer Lt. Gaetano Faillace; public domain}
		\end{column}
		\begin{column}{0.39\textwidth}
			\begin{itemize}
				\item Shouwa = Erleuchteter Frieden, Regierungsdevise
				\pause
				\item 29. April 1901 - 7. Januar 1989
				\item Ten’nou ab 1926
			\end{itemize}
		\end{column}
	\end{columns}
\end{frame}


\begin{frame}{Shouwa-Ten’nou nach dem Krieg}
	\begin{itemize}
		\item überlebte den Krieg
		\pause
		\item sehr beliebt in der Bevölkerung
		\pause
		\item wurde nicht angeklagt
		\item als entmachteter Kaiser im Amt belassen
	\end{itemize}
\end{frame}

\section{Besatzung, Teilung und Gebeitsabtritte}

\subsection{Deutschland}

\begin{frame}{Besatzung}
	\begin{columns}
		\begin{column}{0.6\textwidth}
			\includegraphicscopyright[height = 6cm]
			{Map-Germany-1947.pdf}
			{Wikimedia Commons user 52 Pickup; CC-BY-SA 2.5; Based on map data
			of the IEG-Maps project (Andreas Kunz, B. Johnen and Joachim Robert
			Moeschl: University of Mainz) -- www.ieg-maps.uni-mainz.de}
		\end{column}
		\begin{column}{0.39\textwidth}
			\begin{itemize}
				\item besetzt von USA, Frankreich, UK, UdSSR
				\item regiert direkt durch die Alliierten
			\end{itemize}
		\end{column}
	\end{columns}
\end{frame}

\begin{frame}{Teilung}
	\begin{columns}
		\begin{column}{0.6\textwidth}
			\includegraphicscopyright[height = 6cm]
			{Deutschland_Bundeslaender_1957.png}
			{Wikimedia Commoms User WikiNight; CC-BY-SA 3.0/GFDL 1.2}
		\end{column}
		\begin{column}{0.39\textwidth}
			\begin{itemize}
				\item geteilt in BRD und DDR
				\pause
				\item Berlin hatte Sonderstatus
			\end{itemize}
		\end{column}
	\end{columns}
\end{frame}

\begin{frame}{Gebietsabtritte}
	\begin{columns}
		\begin{column}{0.6\textwidth}
			\includegraphicscopyright[height = 6cm]
			{Map-Germany-1947.pdf}
			{Based on map data of the IEG-Maps project (Andreas Kunz, B. Johnen
			and Joachim Robert Moeschl: University of Mainz) --
			www.ieg-maps.uni-mainz.de; Wikimedia Commons User 52 Pickup}
		\end{column}
		\begin{column}{0.39\textwidth}
			\begin{itemize}
				\item Ostgebiete an Polen und UdSSR
				\pause
				\item Saarland zunächst selbstständig
				\item … dann zur BRD
			\end{itemize}
		\end{column}
	\end{columns}
\end{frame}

\subsection{Japan}

\begin{frame}{Besatzung und Teilung}
	\begin{itemize}
		\item Hauptgebiet nur von den USA besetzt
		\pause
		\item daher auch keine Teilung
		\pause
		\item regiert indirekt durch japanische Beamte
	\end{itemize}
\end{frame}

\begin{frame}{Gebietsabtritte}
	\begin{columns}
		\begin{column}{0.6\textwidth}
			\includegraphicscopyright[height = 6cm]
			{Sea_of_Okhotsk_map.png}
			{CC-BY-SA 3.0/GFDL 1.2; Wikimedia Commons User NormanEinstein}
		\end{column}
		\begin{column}{0.39\textwidth}
			\begin{itemize}
				\item Sachalin an die UdSSR
				\pause
				\item andauernder Konflikt um Kurilen
				\pause
				\item Provinz Chousen (Korea) an USA, UK und UdSSR
			\end{itemize}
		\end{column}
	\end{columns}
\end{frame}

\section{Atombombe}

\subsection{Deutschland}
\begin{frame}
	\begin{itemize}
		\item kein Atombombeneinsatz
		\pause
		\item Folgen nicht ausdenkbar
	\end{itemize}
\end{frame}

\subsection{Japan}
\begin{frame}
	\begin{columns}
		\begin{column}{0.6\textwidth}
			\includegraphicscopyright[height = 6cm]
			{Atomic_bomb_1945_mission_map.pdf}
			{public domain; Wikimedia Commons User Mr.98}
		\end{column}
		\begin{column}{0.39\textwidth}
			\begin{itemize}
				\item Abwurf auf Hiroshima und Nagasaki
				\pause
				\item Hauptauslöser der Kapitulation
			\end{itemize}
		\end{column}
	\end{columns}
\end{frame}

\begin{frame}{Hiroshima}
	\begin{itemize}
		\item inkl. Spätfolgen ca. 90.000 bis 166.000 Opfer
		\pause
		\item 70.000 bis 76.000 zerstörte/beschädigte Häuser
		\pause
		\item Atompilz bis 13\,km Höhe
	\end{itemize}
\end{frame}

\begin{frame}{Nagasaki}
	\begin{itemize}
		\item ursprünglich nicht in Betracht gezogen
		\item gedacht war Kyouto, aber als kulturelles Zentrum durch
			Kriegsminister 	ausgeschlossen
	\end{itemize}
\end{frame}

\begin{frame}{Hibakusha -- Opfer}
	\begin{itemize}
		\item Hibakusha = „Explosionsopfer“
		\pause
		\item Drei Kategorien:
			\pause \begin{itemize}
			\item direkter Tod
			\pause
			\item Tod durch Strahlenkrankheit bis Ende 1945
			\pause
			\item eigentliche Hibakusha: Spätfolgen (z.\,B. Krebs)
			\pause
		\end{itemize}
		\item auch heute Opfer von Diskrimination
	\end{itemize}
\end{frame}


\end{document}