\documentclass{scrartcl}

\usepackage{polyglossia}
\setmainlanguage[spelling = new, babelshorthands = true]{german}

\usepackage{fontspec}
\setmainfont{CMU Serif}

\usepackage{hyperref}

\title{Japans Entwicklung nach dem zweiten Weltkrieg im Vergleich mit
	Deutschland}
\author{Thomas Schneider}
\date{4.\ Dezember 2014}

\begin{document}
\maketitle

\section{Herrscher}
\begin{tabular}{p{0.5\textwidth} p{0.5\textwidth}}
Adolf Hitler & Kaiser Hirohito -- Shouwa-Ten’nou\\
\hline \hline
Führer und Reichskanzler & Shouwa = Erleuchteter Frieden (Regierungsdevise)\\
& Ten’nou = „Himmlischer Herrscher“, Kaiser\\
\hline
20.\  April 1889 -- 30.\ April 1945 & 29.\ April 1901 -- 7.\ Januar 1989\\
Reichskanzler seit 1933 & Ten’nou ab 1926\\
\hline
Selbstmord vor Kriegsende & überlebt\\
wäre angeklagt worden \newline Todesurteil oder lange Haftstrafe & nicht
angeklagt\\
garantiert nicht weiter Staatsoberhaupt & entmachtet im Amt belassen\\
\end{tabular}

\section{Besatzung, Teilung und Gebietsabtritte}
\begin{tabular}{p{0.5\textwidth} p{0.5\textwidth}}
Deutschland & Japan\\
\hline \hline
besetzt von USA, Frankreich, UK, UdSSR & Hauptgebiet nur von USA besetzt\\
regiert direkt durch die Alliierten & regiert indirekt durch japanische Beamte\\
\hline
geteilt in BRD und DDR & nicht geteilt, da nur von USA besetzt\\
Berlin: Sonderstatus &\\
\hline
Ostgebiete an Polen und UdSSR & Sachalin an UdSSR\\
Saarland zunächst selbstständig, dann zu BRD & andauernder Konflikt um Kurilen\\
& Provinz Chousen (Korea) an USA, UK und UdSSR\\
\end{tabular}

\section{Atombombe}
\subsection{Deutschland}
Auf Deutschland wurde nie eine Atombombe abgeworfen, obwohl es bei
Kriegsfortsatz möglicherweise dazu gekommen wäre.  Über die möglichen Folgen
kann man heute nur spekulieren, sie sind für uns eigentlich nicht ausdenkbar.

\subsection{Japan}
\subsubsection{Abwurforte}
\begin{itemize}
	\item Hiroshima
		\begin{itemize}
			\item inkl.\ Spätfolgen ca.\ 90.000 bis 166.000 Opfer
			\item 70.000 bis 76.000 zerstörte/beschädigte Häuser
			\item Atompilz bis 13\,km Höhe
		\end{itemize}
	\item Nagasaki
		\begin{itemize}
			\item ursprünglich nicht in Betracht gezogen
			\item gedacht war Kyouto, aber als kulturelles Zentrum durch
				Kriegsminister ausgeschlossen
		\end{itemize}
\end{itemize}

\subsubsection{Hibakusha -- Opfer}
\begin{itemize}
	\item Hibakusha = „Explosionsopfer“
	\item Drei Kategorien:
		\begin{enumerate}
			\item direkter Tod
			\item Tod durch Strahlenkrankheit bis Ende 1945
			\item eigentliche Hibakusha: Spätfolgen (z.\,B. Krebs)
		\end{enumerate}
	\item auch heute Opfer von Diskrimination
\end{itemize}

\section{Quellen, Slides etc.}
Siehe \url{https://github.com/qsuscs/geschichte_KGL}.

\end{document}
